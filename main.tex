\documentclass[12pt]{article}
\usepackage[utf8]{inputenc}
\usepackage[T1]{fontenc}
\usepackage{geometry}
\geometry{margin=1in}
\usepackage{hyperref}
\usepackage{enumitem}       
\usepackage{apacite}
\usepackage{csquotes}

\title{GDD (Game Design Document) for A2}
\author{Dylan Rumble-Smith}

\setlength{\parindent}{0pt}
\setlength{\parskip}{1em}

\begin{document}
	{\setlength{\parskip}{0pt}%
		\maketitle
		\bibliographystyle{apacite}
		\pagebreak
		\tableofcontents
		\pagebreak
		\bibliography{references}}
		
	\newpage
	
	\section{Introduction}
	In this document, I aim to explain the processes and logic behind the artifacts for my A2 project, while also discussing the culture and history of other games and genres. Additionally, I'll review video game laws and how developers and publishers adapted to them when releasing games.
	
	This document is to be used along with a copy of my video game, as this document will refer to mechanics within that game and explain the context and inner-workings of them.
	
	My video game in question is called Fringe, stylised as FRINGE in promotional material. The title of my game is directly related to its definition, as the word “fringe” is defined as:
	\begin{displaycquote}{fringeMirriamWebster}
		something that is marginal, additional, or secondary to some activity, process, or subject
	\end{displaycquote}
	
	One reason I picked this name is the similarity to the word fling, which promptly explains that within my video game, Fringe, the player is expected to fling themselves off walls and other surfaces within a region nearby them.
	
\end{document}
